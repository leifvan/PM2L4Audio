\documentclass[12pt]{article}
\usepackage{dsfont}
\usepackage{textcomp}
\usepackage{amsmath}
\usepackage{amssymb}
\usepackage{graphicx}


\begin{document}


\title{Solutions to Sheet 6}
\author{Lukas Drexler, Leif Van Holland \\ \\
\textsc{Pattern Matching and Machine Learning} \\
\textsc{for Audio Signal Processing}}
\maketitle

\section*{Exercise 6.1}
Let $x,y:\mathbb{Z}\to\mathbb{C}$ denote signals with the corresponding $z$-transforms $X,Y$. The $z$-transform is defined as $X(z)=\sum_{n\in\mathbb{Z}}x(n)z^{-n}$.
\begin{itemize}
    \item [a)] \textbf{Delay Property.} From $y=x^k=T^k[x]$ we get, that $y(n)=x(n-k)$.
    \begin{align*}
        Y(z):&=\sum_{n\in\mathbb{Z}}y(n)z^{-n}
        =\sum_{n\in\mathbb{Z}}x(n-k)z^{-n}
        \overset{(*)}{=}\sum_{n\in\mathbb{Z}}x(n)z^{-n-k}\\
        &=z^{-k}\sum_{n\in\mathbb{Z}}x(n)z^{-n}
        =z^{-k}X(z).
    \end{align*}
    ${\scriptstyle(*)}$ Shift index of the sum: $n'=n+k$.
    
    \item [b)] \textbf{Multiplication by exponential.} $0\neq a\in\mathbb{C}, y(n)=a^n\cdot x(n)$.
    \begin{align*}
        Y(z):&=\sum_{n\in\mathbb{Z}}y(n)z^{-n}
        =\sum_{n\in\mathbb{Z}}a^n x(n) z^{-n}
        =\sum_{n\in\mathbb{Z}}x(n)\frac{a^n}{z^n}\\
        &=\sum_{n\in\mathbb{Z}}x(n)\left(\frac{z}{a}\right)^{-n}
        =X\left(\frac{z}{a}\right).
    \end{align*}
    
    \item [c)] \textbf{Time-reversal.} Let $y(n)=x(-n)$.
    \begin{align*}
        Y(z):&=\sum_{n\in\mathbb{Z}}y(n)z^{-n}
        =\sum_{n\in\mathbb{Z}}x(-n)z^{-n}
        \overset{(*)}{=}\sum_{n\in\mathbb{Z}}x(n)z^n\\
        &=\sum_{n\in\mathbb{Z}}x(n)\left(\frac{1}{z}\right)^{-n}=X(z^{-1}).
    \end{align*}
    ${\scriptstyle(*)}$ Reverse index of the sum: $n'=-n$.
    
\section*{Exercise 6.2}
The frequency response $H(\omega)$ of
\[ h(n) = \begin{cases}
            \frac{1}{3}, &n=0,1,2,\\
            0, &else.
          \end{cases} \]
is given by
\begin{align*}
    H(\omega):&=\sum_{k\in\mathbb{Z}}h(k)e^{-2\pi i \omega k}\\
    &=h(0)\cdot e^{-2\pi i \omega 0} + h(1)\cdot e^{-2\pi i \omega\cdot 1} + h(2)\cdot e^{-2\pi i \omega \cdot 2}\\
    &=\frac{1}{3}\left(1+e^{-2\pi i \omega}+e^{-4\pi i \omega}\right).
\end{align*}
    
\end{itemize}


\section*{Exercise 6.3}
Let $\omega \neq 0$, $k > 0$ and $x$ be any signal. Then
\begin{align*}
E_{\omega}\left[x^{k} \right](n) = e^{-2\pi i\omega n}x^{k}(n) = e^{-2\pi i\omega n} x(n-k).
\end{align*}
But on the other hand
\begin{align*}
E_{\omega}\left[x \right]^{k}(n) = E_{\omega}[x](n-k) = e^{-2\pi i \omega(n-k)}x(n-k) \neq E_{\omega}\left[x^{k} \right](n).
\end{align*}
Therefore, the operator is not time-invariant for $\omega \neq 0$.


\end{document}
